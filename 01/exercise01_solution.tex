\documentclass[11pt,a4paper]{article}
\usepackage[a4paper, margin=1.3in]{geometry}
\usepackage{mathtools}
\usepackage{fancyhdr}
\usepackage{mathrsfs}

\newcommand{\sheetNr}{1}

\pagestyle{fancy}
\fancyhf{}
\lhead{Game Theory}
\rhead{Exercise Sheet \sheetNr}
\lfoot{Tarek Saier, \today}
\rfoot{Page \thepage\ of \pageref{lastpage}}
\renewcommand{\headrulewidth}{0.3pt}
\renewcommand{\footrulewidth}{0.3pt}
\setlength\parindent{0pt}
\newcommand{\h}[0]{\text{--}}

\begin{document}
\begin{center}
\Huge{\textbf{Game Theory}}\\
\LARGE{\textbf{Exercise Sheet \sheetNr}}
\end{center}
\vspace{2cm}
\begin{tabular}{ll}
Date: & \today\\
Student: & Tarek Saier
\end{tabular}

\section*{Exercise 1.1}
\textbf{(a)}\\
$G=\langle N, (A_i), (u_i)\rangle$ with\\
$N=\{1,2\}$\\
$A_1=A_2=\{u,l,m\}$\hphantom{tabtab}//Note: $u=$upper path, $l=$lower path\\
\hphantom{tabtabtabtabtabtabtabtabtabtabta}$m$=path using vertical edge in the middle\\

\begin{tabular}{c|c|c|c|}
  & $u$ & $l$ & $m$\\
\hline
$u$ & -2.2 , -2.2 & -1.7 , -1.7 & -2.2 , -1.6\\
\hline
$l$ & -1.7 , -1.7 & -2.2 , -2.2 & -2.2 , -1.6\\
\hline
$m$ & -1.6 , -2.2 & -1.6 , -2.2 & -2.1 , -2.1\\
\hline
\end{tabular}\\
\\
\\
stricktly dominated actions:\\
$u_1(a_{-1},m_1)>u_1(a_{-1},l_1)$\\
$u_2(a_{-2},m_2)>u_2(a_{-2},l_2)$\\
$u_1(a_{-1},m_1)>u_1(a_{-1},u_1)$\\
$u_2(a_{-2},m_2)>u_2(a_{-2},u_2)$\\
\\
weakly dominated actions:\\
$u_1(a_{-1},m_1)\ge u_1(a_{-1},l_1)$\\
$u_2(a_{-2},m_2)\ge u_2(a_{-2},l_2)$\\
$u_1(a_{-1},m_1)\ge u_1(a_{-1},u_1)$\\
$u_2(a_{-2},m_2)\ge u_2(a_{-2},u_2)$\\
\\
Nash equilibria: $(m,m)$\\
\\
\textbf{(b)} A notable difference to the lecture example is that fact that the main diagonal of the matrix does not have the same values for all action sets. If both players choose $m$ they gain a higher utility compared to both choosing $u$ or $l$.\\
In both variants adding more players to the game would increase the benefit of taking a $\frac{n_i}{n}$-path alone whilst the rest of the players take the respective other path.
\newpage
\section*{Exercise 1.2}
\textbf{(a)} Nash equilibria: (yield,claim), (claim,yield)\\
The game is \emph{not} strictly competitive, since $\forall a\in A: u_1(a)=-u_2(a)$ does not hold.\\
\\
\textbf{(b)} Nash equilibria: (landside,landside), (seaside,seaside)\\
The game is different insofar, als choosing the \emph{same} action as the opponent is beneficial. For the claim-yield game it's the opposite: playing the action \emph{different} from the opponent's is beneficial.

\label{lastpage}
\end{document}
